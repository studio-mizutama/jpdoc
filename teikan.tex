\documentclass[11pt,a4paper]{jsarticle}
%
\usepackage{bm}
\usepackage{graphicx}
%
\setlength{\textwidth}{\fullwidth}
\setlength{\textheight}{39\baselineskip}
\addtolength{\textheight}{\topskip}
\setlength{\voffset}{-0.5in}
\setlength{\headsep}{0.3in}
%\pagestyle{fancy}

%\lhead[]{}
%\rhead[株式会社科学計算総合研究所]{株式会社科学計算総合研究所}

\renewcommand{\presectionname}{第}
\renewcommand{\postsectionname}{章}
\renewcommand{\labelenumi}{第\arabic{enumi}条}

\makeatletter
\def\section{\@startsection{section}{1}{-40pt}{15pt}{10pt}{\normalfont\Large\bfseries\centering}}

\setlength{\labelsep}{10pt}

\newcommand{\header}[1]{\vspace{5pt}\hspace{-42pt}({#1})}

\begin{document}
  
\begin{center}
{\bf {\LARGE 株式会社科学計算総合研究所 定款}}
\end{center}

\begin{enumerate}

\item[]

\begin{center}
\section{総則}
\end{center}


\header{商号}
\item 当会社は,株式会社科学計算総合研究所と称する。英文では,Research Institute for Computational Science Co. Ltd. と表示する。

\header{目的}
\item 当会社は,次の事業を営むことを目的とする。
\begin{enumerate}
	\item[1.] 計算科学,計算機科学及び応用力学の分野における研究並びに知的財産の創出
	\item[2.] 計算資源の管理,貸与,買取及び販売
	\item[3.] 計算機の設計,開発,販売,買取,保守及び輸出入
	\item[4.] ソフトウェアの開発,販売,買取,保守及び輸出入
	\item[5.] コンピュータシミュレーションを活用した解析の請負
	\item[6.] その他適法な一切の事業
\end{enumerate}

\header{本店の所在地}
\item 当会社は,本店を東京都文京区に置く。

\header{公告の方法}
\item 当会社の公告は,電子公告とする。ただし,事故その他やむを得ない事由によって電子公告による公告をすることができない場合は,官報に掲載してする。


\section{株式}


\header{発行する株式の総数}
\item 当会社の発行する株式の総数は,1万株とする。

\header{株式の譲渡制限}
\item 当会社の株式を譲渡するには,取締役会の承認を受けなければならない。

\header{株券の不発行}
\item 当会社は,株券を発行しない。

\header{名義書換}
\item 株式取得者が株主名簿記載事項を株主名簿に記載又は記録するには,当会社所定の書式による請求書に,その取得した株式の株主として株主名簿に記載又は記録された者又はその相続人その他の一般承継人及び株式取得者が署名又は記名押印し共同して請求しなければならない。

\header{質権の登録及び信託財産の表示}
\item 当会社の株式について質権の登録又は信託財産の表示を請求するには,当会社所定の書式による請求書に当事者が署名又は記名押印し,共同して請求しなければならない。その登録又は表示の抹消についても同様とする。

\header{手数料}
\item 前二条に定める請求をする場合には,当会社所定の手数料を支払わなければならない。

\header{株主の住所等の届出}
\item 当会社の株主及び登録された質権者又はその法定代理人若しくは代表者は,当会社所定の書式により,その氏名,住所及び印鑑を当会社に届け出なければならない。届出事項に変更を生じたときも,その事項につき,同様とする。

\header{株主名簿の閉鎖及び基準日}
\item 当会社は,営業年度末日の翌日から定時株主総会の終結の日まで株主名簿の記載の変更を停止する。前項の他,株主又は質権者として権利を行使すべき者を確定するため必要があるときは,あらかじめ公告して,一定期間株主名簿の記載の変更を停止し,又は基準日を定めることができる。


\section{株主総会}


\item[]
\header{招集}
\item 当会社の定時株主総会は,営業年度末日の翌日から3ヵ月以内に招集し,臨時株主総会は,必要に応じて招集する。

\header{招集手続きの省略}
\item 株主総会は,その総会において,議決権を行使することができるすべての株主の同意があるときは,招集手続きを経ずに開催することができる。

\header{議長}
\item 株主総会の議長は,代表取締役がこれに当たる。
\item[2] 代表取締役に事故があるときは,取締役会においてあらかじめ定めた順序により他の取締役が議長になる。
\item[3] 取締役全員に事故があるときは,総会において出席株主のうちから議長を選出する。

\header{決議の方法}
\item 株主総会の決議は,法令又は定款に別段の定めがある場合の他,出席した株主の議決権の過半数をもって決する。株主総会の特別決議は,総株主の議決権の3分の1以上を有する株主が出席して,その議決権の3分の2以上をもって決する。

\header{書面による決議}
\item 株主総会の決議の目的である事項について,取締役又は株主から提案があった場合には,その事項につき議決権を行使することができるすべての株主が,書面によってその提案に同意したときは,その提案を可決する総会の決議があったものとみなす。


\section{取締役,取締役会,代表取締役及び監査役}

\item[]
\header{取締役会の設置}
\item 当会社に取締役会を設置する。

\header{監査役の設置}
\item 当会社に監査役を置く。

\header{取締役及び監査役の員数}
\item 当会社の取締役は3名以上とし,監査役は1名以上とする。

\header{取締役及び監査役の選任}
\item 当会社の取締役及び監査役は,株主総会において議決権を行使することができる株主の議決権の数の3分の1以上の議決権を有する株主が出席し,その議決権の過半数の決議によって選任する。取締役の選任については,累積投票によらないものとする。

\header{取締役及び監査役の任期}
\item 取締役及び監査役の任期は,就任後5年内の最終の決算期に関する定時株主総会の終結の時までとする。補欠又は増員により選任された取締役は,他の取締役の任期の残存期間と同一とする。任期の満了前に退任した監査役の補欠として選任された監査役の任期は,退任した監査役の任期が満了すべき時までとする。

\header{代表取締役及び役付取締役}
\item 取締役会は,その決議により取締役の中から代表取締役を1名以上定める。 
\item[2] 取締役会は,その決議により取締役の中から取締役会長,専務取締役及び常務取締役を定めることができる。

\header{報酬及び退職慰労金}
\item 取締役及び監査役の報酬及び退職慰労金はそれぞれ株主総会の決議をもって定める。

\clearpage
\section{計算}

\item[]
\header{事業年度}
\item 当会社の事業年度は,毎年4月1日から翌年の3月31日までの年1期とする。

\header{利益配当}
\item 利益配当金は,毎営業年度末日現在における株主名簿に記載された株主又は質権者に対して支払う。利益配当金が,その支払提供の日から満3年を経過しても受領されないときは,当会社はその支払義務を免れるものとする。



\end{enumerate}

\vspace{20pt}
\begin{flushright}
平成28年5月10日\\
\vspace{10pt}

上記は当会社の現行定款と相違ない.\\
\vspace{10pt}
東京都文京区本郷七丁目3番1号\\
東京大学アントレプレナープラザ\\
株式会社科学計算総合研究所\\
代表取締役 井原遊
\end{flushright}
\end{document}
